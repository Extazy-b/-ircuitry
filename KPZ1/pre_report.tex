\documentclass[12pt,a4paper]{article}

% -----------------------------------
% Подключение необходимых пакетов
% -----------------------------------
\usepackage[utf8]{inputenc}        % Кодировка UTF-8
\usepackage[T2A]{fontenc}          % Поддержка кириллицы
\usepackage[russian]{babel}        % Русский язык

\usepackage{amsmath, amssymb}      % Математические символы
\usepackage{geometry}              % Настройка полей страницы
\usepackage{graphicx}              % Вставка изображений
\usepackage{float}                 % Управление положением картинок
\usepackage{caption}               % Подписи к рисункам
\usepackage{circuitikz}            % Рисование электрических схем
\usepackage{tikz}
\usepackage{hyperref}              % Ссылки внутри документа

\usepackage{pgfplots}
\pgfplotsset{compat=1.18}

% Настройка полей страницы
\geometry{top=2cm,bottom=2cm,left=2cm,right=2cm}

\title{КПЗ №1}
\author{Есиков Сергей Дмитриевич и Иванова Алиса Игоревна}
\date{\today}

\begin{document}
	\maketitle
	\subsection*{Задача\newline}
	
	Заданы параметры катушки (L, RL) и частота резонанса.
	
	Необходимо:
	\begin{enumerate}
		\item построить последовательный колебательный контур;
		\item исследовать его свойства в частотной области и в координатах времени;
		\item сопоставить данные анализа в координатах частоты, и времени;
		\item исследовать влияние внутреннего сопротивления источника энергии
		на колебательные процессы в контуре.
	\end{enumerate}

	\subsection*{Условие\newline}
	
	Был выбран вариант № 16, по которому контур имеет следующие параметры
	\begin{itemize}
		\item $L$ (индуктивность катушки) -- 470 мкГн = $4.7 \cdot 10^{-4} H$
		\item $R_L$  ( сопротивление потерь катушки) -- 4 Ом
		\item $f_0$ (резонансная частота контура) -- 32 кГц = $3.2 \cdot 10^{3} Hz$
	\end{itemize}
		
	\subsection*{Схема\newline}
	\begin{figure}[H]
		\centering
		\begin{circuitikz}[scale=1.5, european]
			 % цепь
			\draw (0, 0) to[isourceAM, l=$e$] (0, 2) to[R=$R_g$] (2, 2) to[R=$R_L$] (4, 2) to[cute inductor, l=$L$] (4, 0) to[C=$C$] (2, 0) to[R=$R_C$] (0, 0);
			%стрелка напряжения
			\draw[->] (0.5, 1.5) -- (0.5, 0.5);
			\draw (0.5, 1) node[right] {$u_e$};
			%стрелка тока
			\draw[-latex] (2.5, 0.5) -- (2.1, 0.5);
			\draw [rounded corners=2mm] (1.5, 0.5) rectangle (3.5, 1.5);
			\draw (2.4, 0.5) node[above] {i};
		\end{circuitikz}
	\end{figure}
	
	\newpage
	\section{Выбор конденсатора.}
	Вычислите ёмкость конденсатора, обеспечивающую резонанс на заданной частоте. Руководствуясь рядом $Е24$, выберите ёмкость, подходящую для конденсатора, обеспечивающего резонанс $LC$-контура на частоте ближайшей заданной. Будем считать, что сопротивления потерь конденсатора и катушки одинаковые: $R_C$ = $R_L$.
	
	\subsubsection*{Резонансная частота колебательного контура}
	
		\[f_0 = \frac{1}{2\pi\sqrt{LC}}\]
	
	\subsubsection*{Необходимая ёмкость конденсатора}
	
		\[C = \frac{1}{4\pi^2 f^2 L}\]
		
	\subsubsection*{Подставляя}
	
		\[C = \frac{1}{4\pi^2 32^2 \cdot 47} \approx 5.263 \cdot 10^{-7} F = 52.63 n F\]
	
	\subsubsection*{Подходящие из E24 ряда номиналы}
		
		\[51 nF, 56 nF\]
		
	\subsubsection*{Приблжение снизу к резонансной частоте означает приближение сверху к ёмкости конденсатора}
	
	\newpage
	
	\section{Параметры контура.}
	Для колебательного контура с заданной индуктивностью и выбранной ёмкостью вычислите частоту резонанса, добротность контура, ширину резонансной
	характеристики тока и её отношение к частоте резонанса (сопоставьте с 1⁄Q).
	
	(Пока не включайте в состав контура сопротивление источника $R_g$.)
	
	\subsubsection*{Резонансная частота для $56 nF$}
		
		\[f^`_0 = \frac{1}{2\pi\sqrt{4.7 \cdot 5.6 \cdot 10^{-12}}} \approx 31.022 kHz\]
		
	\subsubsection*{Погрешность}
		
		\[1 - \frac{31022}{32000} \approx 3.05\% \]
		
	\subsubsection*{Добротность конутра}
	
		\[Q = \frac{\sqrt{L/C}}{2R_L} = \frac{\sqrt{4.7 \cdot 10^{4} / 5.6}}{8} \approx 11.452\]
		
	\subsubsection*{Ширина резонансной характеристики тока}
	
		\[\Delta f = \frac{f_0}{Q} = \frac{\frac{\omega_0}{2\pi}}{\frac{\omega_0 L}{R}} = \frac{R}{2\pi L} = \frac{8}{2 \cdot 4.7 \pi\cdot 10^{-4}} \approx 2.709 kHz\]
		
	\subsubsection*{Отношение $\frac{\Delta f}{f_0}$}
	
		\[\frac{\Delta f}{f_0} = \frac{2709}{31022} = 0.0873\]
	
	\newpage	
		
	\section{Модель.}
	Постройте Multisim-модель анализа частотных и временных свойств последовательного контура.
	
\end{document}