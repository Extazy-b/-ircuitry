\documentclass[12pt,a4paper]{article}

% -----------------------------------
% Подключение необходимых пакетов
% -----------------------------------
\usepackage[utf8]{inputenc}        % Кодировка UTF-8
\usepackage[T2A]{fontenc}          % Поддержка кириллицы
\usepackage[russian]{babel}        % Русский язык

\usepackage{amsmath, amssymb}      % Математические символы
\usepackage{geometry}              % Настройка полей страницы
\usepackage{graphicx}              % Вставка изображений
\usepackage{float}                 % Управление положением картинок
\usepackage{caption}               % Подписи к рисункам
\usepackage{circuitikz}            % Рисование электрических схем
\usepackage{hyperref}              % Ссылки внутри документа

\usepackage{pgfplots}
\pgfplotsset{compat=1.18}

% Настройка полей страницы
\geometry{top=2cm,bottom=2cm,left=2cm,right=2cm}

\title{Домашнее задание №1}
\author{Есиков Сергей Дмитриевич}
\date{07.09.2025}

\begin{document}
	\maketitle
	\section*{Мостовые цепи.}
	\section*{Задача 1}
		\begin{figure}[H]
			\centering
			\begin{circuitikz}[scale=1.5, american]
				% Узлы
				\coordinate (C) at (0,-1); 
				\coordinate (B) at (1,-2); 
				\coordinate (A) at (1,0);  
				\coordinate (D) at (2,-1);  
				\coordinate (Oc) at (0.6,-1); 
				\coordinate (Od) at (1.3,-1);  
				\coordinate (Va) at (-0.5,0); 
				\coordinate (Vb) at (-0.5,-2);  
				
				% Подписи узлов
				\node[above] at (Oc) {c};
				\node[above] at (Od) {d};
				\node[left] at (Va) {a};
				\node[left] at (Vb) {b};
				
				% Резисторы
				\draw (C) to[R=$R_1$] (A);
				\draw (B) to[R=$R_2$] (C);
				\draw (D) to[R=$R_4$] (B);
				\draw (A) to[R=$R_3$] (D);
				
				% Источник
				\draw (Vb) to[isource,l=$V_s$] (Va);
				\draw (Va) to[short] (A);
				\draw (Vb) to[short] (B);
		
				% Съем напряжения
				\draw (Oc) to[short, -*] (C);
				\draw (Od) to[short, -*] (D);
			\end{circuitikz}
		\end{figure}
		При каком соотношении сопротивлений резисторов мост Уитстона
		сбалансирован?

	\subsection*{Решение}
	Будем искать решение через анализ потенциалов во всех узлах
	
	$U_{a} = 0$
	
	$U_{b} = U_{s}$	
	
	2. Мост Уитсона сбалансирован, значит напряжение между узлами c и d равно нулю:
	
	$U_{cd} = 0$
	
	$U_{cd} = U_{d} - U_{c}$

	3. По первому з-ну Кирхгофа:
	
	$0 = I_{d} = i_{ad} + i_{bd} = \frac{U_{ad}}{R_3} + \frac{U_{bd}}{R_4}$
	
	$0 = \frac{U_d - U_a}{R_3} + \frac{U_d - U_b}{R_4}$
	
	$0 = \frac{U_d}{R_3} + \frac{U_d - U_s}{R_4}$
	
	$\frac{U_s}{R_4} = U_d \cdot \left(\frac{1}{R_4} + \frac{1}{R_3}\right) = $
	$U_{d} \cdot \frac{R_3 + R_4}{R_3 R_4}$
	
	$U_{d} = \frac{U_s R_3 R_4}{R_4 \left(R_3 + R_4\right)} = U_s \cdot \frac{R_3}{R_3 + R_4}$
	
	4. Аналогично
	
	$U_c = U_s \cdot \frac{R_1}{R_1 + R_2}$
	
	5. Тогда
	
	$U_{cd} = U_s \cdot \left[\frac{R_3}{R_3 + R_4} - \frac{R_1}{R_1 + R_2}\right] = U_s \cdot \frac{R_2 R_3 - R_1 R_4}{\left(R_3 + R_4\right) \cdot \left(R_1 + R_2\right)}$
	
	Получаем
	
	$U_{cd} = 0 \Leftrightarrow R_2 R_3 = R_1 R_4 \Leftrightarrow \frac{R_1}{R_2} = \frac{R_3}{R_4}$
	

	\section*{Задача 2}
			\begin{figure}[H]
		\centering
		\begin{circuitikz}[scale=1.5, american]
			% Узлы
			\coordinate (C) at (0,-1); 
			\coordinate (B) at (1,-2); 
			\coordinate (A) at (1,0);  
			\coordinate (D) at (2,-1);  
			\coordinate (Oc) at (0.6,-1); 
			\coordinate (Od) at (1.3,-1);  
			\coordinate (Va) at (-0.5,0); 
			\coordinate (Vb) at (-0.5,-2);  
			
			% Подписи узлов
			\node[above] at (Oc) {c};
			\node[above] at (Od) {d};
			\node[left] at (Va) {a};
			\node[left] at (Vb) {b};
			
			% Резисторы
			\draw (C) to[R=$R$] (A);
			\draw (B) to[R=$R + \Delta R$] (C);
			\draw (D) to[R=$R$] (B);
			\draw (A) to[R=$R$] (D);
			
			% Источник
			\draw (Vb) to[isource,l=$V_s$] (Va);
			\draw (Va) to[short] (A);
			\draw (Vb) to[short] (B);
			
			% Съем напряжения
			\draw (Oc) to[short, -*] (C);
			\draw (Od) to[short, -*] (D);
		\end{circuitikz}
		\begin{circuitikz}[scale=1.5, american]
			% Узлы
			\coordinate (C) at (0,-1); 
			\coordinate (B) at (1,-2); 
			\coordinate (A) at (1,0);  
			\coordinate (D) at (2,-1);  
			\coordinate (Oc) at (0.6,-1); 
			\coordinate (Od) at (1.3,-1);  
			\coordinate (Va) at (-0.5,0); 
			\coordinate (Vb) at (-0.5,-2);  
			
			% Подписи узлов
			\node[above] at (Oc) {c};
			\node[above] at (Od) {d};
			\node[left] at (Va) {a};
			\node[left] at (Vb) {b};
			
			% Резисторы
			\draw (C) to[R=$R$] (A);
			\draw (B) to[R=$R + \Delta R$] (C);
			\draw (D) to[R=$R$] (B);
			\draw (A) to[R=$R + \Delta R	$] (D);
			
			% Источник
			\draw (Vb) to[isource,l=$V_s$] (Va);
			\draw (Va) to[short] (A);
			\draw (Vb) to[short] (B);
			
			% Съем напряжения
			\draw (Oc) to[short, -*] (C);
			\draw (Od) to[short, -*] (D);
		\end{circuitikz}
	\end{figure}
	Мост Уитстона питают стабилизированным напряжением V.
	Получите и сравните формулы, связывающие напряжение U на диагонали моста с приращением сопротивления $\Delta R$ в нижнем плече моста и в
	другом случае – для моста с одинаковыми приращениями сопротивлений $\Delta R$ в нижнем и верхнем плечах моста.
	
	\subsection*{Решение}
	Пользуясь выводом прошлой задачи, подставим
	
	В первом случае:
	
	$U_{cd} = U_s \cdot \frac{\left(R + \Delta R\right) R - R^2}{2R \cdot \left(2R + \Delta R\right)} = U_s \cdot \frac{R \Delta R}{4R^2 + 2R \Delta R}$
	
	Во втором случае:
	
	$U_{cd} = U_s \cdot \frac{\left(R + \Delta R\right)^2 - R^2}{\left(2R + \Delta R\right)^2} 
	= U_s \cdot \left[1 - \left(\frac{R}{2R + \Delta R}\right)^2\right]$
	
	\section*{Задача 3}
	\begin{figure}[H]
		\centering
		\begin{circuitikz}[scale=1.5, american]
			% Узлы
			\coordinate (C) at (0,-1); 
			\coordinate (B) at (1,-2); 
			\coordinate (A) at (1,0);  
			\coordinate (D) at (2,-1);  
			\coordinate (Oc) at (0.6,-1); 
			\coordinate (Od) at (1.3,-1);   
			
			% Подписи узлов
			\node[left] at (C) {c};
			\node[right] at (D) {d};
			\node[above] at (A) {a};
			\node[below] at (B) {b};
			
			% Резисторы
			\draw (C) to[R=$R$] (A);
			\draw (B) to[R=$2R$] (C);
			\draw (D) to[R=$8R$] (B);
			\draw (A) to[R=$4R$] (D);
			\draw (C) to[R=$R$] (D);
			
		\end{circuitikz}
	\end{figure}
	Вычислите сопротивление на входе мостовой схемы.
	\subsection*{Решение}
	Для упрощения вычислений прибегнем к матричной форме законов Кирхгофа
	\begin{itemize}
		\item $V$ — вектор узловых потенциалов относительно выбранного опорного узла (земли);
		\item $A$ — матрица проводимостей (составленная из сопротивлений сети);
		\item $I$ — вектор внешних токов, поступающих в узлы от источников.
	\end{itemize}
	
	Элементы матрицы $A = (a_{ij})$ определяются следующим образом:
	\[
	a_{ij} =
	\begin{cases}
		\displaystyle \sum\limits_{k \neq i} g_{ik}, & \text{если } i=j, \\[1.5ex]
		\displaystyle -g_{ij}, & \text{если } i \neq j \text{ и узлы $i$ и $j$ соединены резистором с проводимостью } g_{ij}, \\[1.5ex]
		0, & \text{если соединения между $i$ и $j$ нет.}
	\end{cases}
	\]
	
	Закон Кирхгофа для токов (KCL) в узловой форме записывается так:
	\[
	A V = I .
	\]
	
	Применяя это на вышеизложенную схему, получаем:
	\[
	\begin{bmatrix}
		\frac{1}{R_{ac}} + \frac{1}{R_{bc}} + \frac{1}{R_{cd}} & -\frac{1}{R_{cd}} \\[2ex]
		-\frac{1}{R_{cd}} & \frac{1}{R_{ad}} + \frac{1}{R_{bd}} + \frac{1}{R_{cd}}
	\end{bmatrix}
	\begin{bmatrix} V_c \\ V_d \end{bmatrix}
	=
	\begin{bmatrix}
		\frac{V_a}{R_{ac}} + \frac{V_b}{R_{bc}} \\[2ex]
		\frac{V_a}{R_{ad}} + \frac{V_b}{R_{bd}}
	\end{bmatrix}.
	\]
	
	Тогда:
	\[V = \frac{1}{\det(A)}
	\begin{bmatrix}
		\dfrac{1}{R_{ad}} + \dfrac{1}{R_{bd}} + \dfrac{1}{R_{cd}} & \dfrac{1}{R_{cd}} \\[2ex]
		\dfrac{1}{R_{cd}} & \dfrac{1}{R_{ac}} + \dfrac{1}{R_{bc}} + \dfrac{1}{R_{cd}}
	\end{bmatrix}
	\begin{bmatrix}
		\dfrac{V_a}{R_{ac}} + \dfrac{V_b}{R_{bc}} \\[2ex]
		\dfrac{V_a}{R_{ad}} + \dfrac{V_b}{R_{bd}}
	\end{bmatrix}
	\], где
	
	\[\det(A) =
	\left(\dfrac{1}{R_{ac}} + \dfrac{1}{R_{bc}} + \dfrac{1}{R_{cd}}\right)
	\left(\dfrac{1}{R_{ad}} + \dfrac{1}{R_{bd}} + \dfrac{1}{R_{cd}}\right)
	- \left(\dfrac{1}{R_{cd}}\right)^2.
	\]
	
	Подставляя все значения, получаем:
	
	\[V = \frac{1}{\det(A)}
	\begin{bmatrix}
		\dfrac{1}{4R} + \dfrac{1}{8R} + \dfrac{1}{R} & \dfrac{1}{R} \\[2ex]
		\dfrac{1}{R} & \dfrac{1}{R} + \dfrac{1}{2R} + \dfrac{1}{R}
	\end{bmatrix}
	\begin{bmatrix}
		\dfrac{V_a}{R} + \dfrac{V_b}{2R} \\[2ex]
		\dfrac{V_a}{4R} + \dfrac{V_b}{8R}
	\end{bmatrix}
	\],где
	
	\[
	\det(A) =
	\left(\dfrac{1}{R} + \dfrac{1}{2R} + \dfrac{1}{R}\right)
	\left(\dfrac{1}{4R} + \dfrac{1}{8R} + \dfrac{1}{R}\right)
	- \left(\dfrac{1}{R}\right)^2.
	\]
	
	Упрощаем:
	
	\[
	\det(A) =
	\dfrac{5}{2R} \cdot \dfrac{11}{8R} - \dfrac{1}{R^2} = 
	\dfrac{39}{16R^2}
	\]
	
	\[
	V = \frac{16R^2}{39}
	\begin{bmatrix}
		\dfrac{11}{8R} & \dfrac{1}{R} \\[2ex]
		\dfrac{1}{R} & \dfrac{5}{2R}
	\end{bmatrix}
	\begin{bmatrix}
		\dfrac{V_a}{R} + \dfrac{V_b}{2R} \\[2ex]
		\dfrac{V_a}{4R} + \dfrac{V_b}{8R}
	\end{bmatrix}
	\]
	
	
	\[
	\boxed{\,V_c = V_d = \frac{2V_a + V_b}{3}\,}
	\]
	
	
	Ток через узел a
	\[
	I_a = \frac{V_a - V_c}{R_{ac}} + \frac{V_a - V_d}{R_{ad}}.
	\]
	\[
	I_a = \frac{V_a - \tfrac{2V_a+V_b}{3}}{R} + \frac{V_a - \tfrac{2V_a+V_b}{3}}{4R}.
	\]
	
	Упрощая:
	
	\[
	I_a = \frac{V_a - V_b}{3}\left(\frac{1}{R}+\frac{1}{4R}\right)
	= \frac{5}{12R}(V_a - V_b).
	\]
	
	Следовательно, эквивалентное сопротивление:
	
	\[
	R_{ab} = \frac{V_a - V_b}{I_a} = \frac{12}{5}R.
	\]
	
	\[
	\boxed{R_{ab} = 2.4R}
	\]
	
	
	\section*{Переходной процесс.}
	\section*{Задача 4}
	\begin{figure}[H]
		\centering
		\begin{circuitikz}[scale=1.6, american]
			% Узлы
			\coordinate (A) at (0,-0.6); 
			\coordinate (B) at (0,0.4); 
			\coordinate (C) at (1,0.4);  
			\coordinate (D) at (1,-0.6); 
			
			% Элементы
			\draw (A) to[isource, l=$U$] (B);
			\draw (B) to[capacitor, C=$C$] (C);
			\draw (C) to[resistor, R=$R$] (D);
			\draw (D) to[short] (A);
		\end{circuitikz}
		\begin{circuitikz}[scale=1.6, american]
			% Узлы
			\coordinate (A) at (0,0); 
			\coordinate (B) at (0,1); 
			\coordinate (C) at (1,1);  
			\coordinate (D) at (1,0); 
			
			% Элементы
			\draw (A) to[isource, l=$U$] (B);
			\draw (B) to[inductor, L=$L$] (C);
			\draw (C) to[resistor, R=$R$] (D);
			\draw (D) to[short] (A);
		\end{circuitikz}
	\end{figure}
	К цепям (см. рисунки) в момент t = 0 подключают источник постоянной
	ЭДС. Начальные условия нулевые: при t < 0 напряжение на емкости, ток через
	индуктивность равны нулю. Постройте качественно графики поведения во
	времени токов и напряжений на элементах.
	
	\subsection*{Решение RC-цепь}
	По з-ну Кирхгофа:
	
	\[
	U=u_C(t)+i(t)R	
	\]
	
	Ток через конденсатор:
	
	\[
	i(t)=C\dfrac{du_C}{dt}
	\]
	
	Получаем:
	\[
	U=u_C(t)+RC\dfrac{du_C}{dt}
	\]
	\[
	\dfrac{1}{RC}u_C(t)+\dfrac{du_C}{dt} = 	\dfrac{U}{RC}
	\]
	
	Решив диф-ур получаем
	\[
	u(t) = U (1 - e^{-t/RC})
	\]
	\[
	i(t)=C\dfrac{du_C}{dt} = \dfrac{U}{R} e^{-t/RC}
	\]
	
	\begin{tikzpicture}
		\begin{axis}[
			width=10cm, height=5cm,
			axis lines=middle,
			xlabel={$t$}, ylabel={$u_C, i$},
			ymin=0, ymax=1.2,
			xtick=\empty, ytick=\empty,
			legend pos=outer north east
			]
			\addplot[blue, thick, domain=0:5, samples=100] {1 - exp(-x)};
			\addlegendentry{$u_C(t)$}
			
			\addplot[red, dashed, thick, domain=0:5, samples=100] {exp(-x)};
			\addlegendentry{$i(t)$}
		\end{axis}
	\end{tikzpicture}	
	\subsection*{RL - цепь}
	
	По з-ну Кирхгофа:
	\[
	U=i(t)R + L\dfrac{i(t)}{dt}
	\]
	
	Получаем аналогично:
	\[
	i(t) = \dfrac{U}{R}(1 - e^{-Rt/L})
	\]
	\[
	u_L(t) = U e^{-Rt/L}
	\]
	
	\begin{tikzpicture}
		\begin{axis}[
			width=10cm, height=5cm,
			axis lines=middle,
			xlabel={$t$}, ylabel={$i, u_L$},
			ymin=0, ymax=1.2,
			xtick=\empty, ytick=\empty,
			legend pos=outer north east
			]
			\addplot[blue, thick, domain=0:5, samples=100] {1 - exp(-x)};
			\addlegendentry{$i(t)$}
			
			\addplot[red, dashed, thick, domain=0:5, samples=100] {exp(-x)};
			\addlegendentry{$u_L(t)$}
		\end{axis}
	\end{tikzpicture}
	\section*{Принцип суперпозиции}
	\section*{Задача 5}
	\begin{figure}[H]
		\centering
		\begin{circuitikz}[american, scale = 1.3]
			\draw (0, 3) to[R=$R_1$] (0, 0);
			\draw[->] (-0.5, 2) -- (-0.5, 1) node[left] {$u_1$};
			
			\draw (4, 0) to[R=$R_4$] (4, 3);
			\draw[->] (4.5, 2) -- (4.5, 1) node[right] {$u_4$};
			
			\draw (2, 2) to[R=$R_2$, above] (0, 3);
			\draw (4, 3) to[R=$R_3$, above] (2, 2);
			
			\draw (4, 0) to[short] (0, 0);
			
			\draw (0, 3) to[isource, l=$i$] (4, 3);
			\draw (2, 2) to[vsource, l_=$e$, 
			bipoles/vsourceam/inner plus=e,
			bipoles/vsourceam/inner minus=0
			] (2, 0);
			
			\draw (2, 0) node[ground] (2, -1) {};
		\end{circuitikz}
	\end{figure}
	
	Выведите формулу для напряжения u4, используя принцип суперпозиции
	
	\subsection*{Решение}
	Рассмотрим вклад каждого источника по отдельности, представив, что второй источник неактивен
	
	\subsection*{Нулевой ток $i = 0$}
		\begin{figure}[H]
		\centering
		\begin{circuitikz}[american, scale = 1]
			\draw (0, 0) to[R=$R_2$] (2, 2);
			\draw (2, 2) to[R=$R_1$] (4, 0);
			\draw (2, -2) to[R=$R_4$] (4, 0);		
			\draw (0, 0) to[R=$R_3$] (2, -2);
			
			\draw (4, 0) -- (4.5, 0) node[above] {0};
			\draw (-0.5, 0) -- (0, 0) node[above] {e};
		\end{circuitikz}
	\end{figure}
	
	По закону Ома:
	
	$U^e_{\text{общ}} = e$
	
	$U^e_{12} = U^e_{34} = U^e_\text{общ} = e$
	
	$R^e_{34} = R_3 + R_4$
	
	$I^e_{34} = \dfrac{U^e_{34}}{R^e_{34}} = \dfrac{e}{R_3 + R_4}$
	
	$I^e_4 = I^e_3 = I^e_{34}$
	
	$U^e_4 = I^e_4 \cdot R_4 = \dfrac{R_4 \cdot e}{R_3 + R_4}$
	
	\[
	\boxed{U^e_4 = \dfrac{R_4 \cdot e}{R_3 + R_4}}
	\]


	\subsection*{Нулевой ЭДС $e = 0$}
	\begin{figure}[H]
		\centering
		\begin{circuitikz}[american, scale = 1.3]
			\draw (-0.5, 0) -- (0, 0) node[above] {0};
			
			\draw (0, 0) to[short] (1, 1) to[R=$R_2$] (2, 1) to[short] (3, 0);
			\draw (0, 0) to[short] (1, -1) to[R=$R_1$] (2, -1) to[short] (3, 0);
			
			\draw (3, 0) to[isource, l=$i$] (4, 0);			
			
			\draw (4, 0) to[short] (5, 1) to[R=$R_3$] (6, 1) to[short] (7, 0);
			\draw (4, 0) to[short] (5, -1) to[R=$R_4$] (6, -1) to[short] (7, 0);
			
			\draw (7, 0) -- (7.5, 0) node[above] {0};
		\end{circuitikz}
	\end{figure}
	
	По закону Кирхгофа:
	
	$I^i_3 + I^i_4 = i$
	
	$\frac{U^i_3}{R_3} + \frac{U^i_4}{R_4} = i$
	
	$i = \frac{U^i_4}{R_3} + \frac{U^i_4}{R_4} = U^i_4 \cdot \frac{R_3 + R_4}{R_3 R_4} $
	
	\[
	\boxed{U^i_4 = \dfrac{i \cdot R_3 \cdot R_4}{R_3 + R_4}}
	\]
	
	
	\subsection*{Результат}
	Суммируя, получаем
	
	$U_4 = U^e_4 + U^i_4 = \dfrac{R_4 \cdot e}{R_3 + R_4} + \dfrac{i \cdot R_3 \cdot R_4}{R_3 + R_4}$
	
	\[
	\boxed{U_4 = \dfrac{R_4 \cdot (e + i R_3)}{R_3 + R_4}}
	\]
	
	\section{Задача 6}
		\begin{figure}[H]
		\centering
		\begin{circuitikz}[american, scale = 2]
			\draw (0, 0) to[isource, l=$e_1$] (0, 2) -- (3, 2) to[isource, l=$e_2$, invert] (3, 0) to[R=$R_4$] (0, 0);
			\draw (0, 0) to[R=$R_1$] (1.5, 1);
			\draw (1.5, 1) to[R=$R_2$] (3, 0);
			\draw (1.5, 1) to[R=$R_3$] (1.5, 2);
		\end{circuitikz}
		\end{figure}
		В схеме $e_1$ = 160 мB, $e_2$ = 100 мB, $R_1$ = $R_2$ = 100 Ом, $R_3$ = $R_4$ = 400 Ом.
		
		Выясните: 
		
		а. значения токов в ветвях; 
		
		б. при каком значении ЭДС $e_1$ нет тока в ветви с $R_4$?
	
\end{document}
