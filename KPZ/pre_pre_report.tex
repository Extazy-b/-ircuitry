\documentclass[12pt,a4paper]{article}

% -----------------------------------
% Подключение необходимых пакетов
% -----------------------------------
\usepackage[utf8]{inputenc}        % Кодировка UTF-8
\usepackage[T2A]{fontenc}          % Поддержка кириллицы
\usepackage[russian]{babel}        % Русский язык

\usepackage{amsmath, amssymb}      % Математические символы
\usepackage{geometry}              % Настройка полей страницы
\usepackage{graphicx}              % Вставка изображений
\usepackage{float}                 % Управление положением картинок
\usepackage{caption}               % Подписи к рисункам
\usepackage{circuitikz}            % Рисование электрических схем
\usepackage{tikz}
\usepackage{hyperref}              % Ссылки внутри документа

\usepackage{pgfplots}
\pgfplotsset{compat=1.18}

% Настройка полей страницы
\geometry{top=2cm,bottom=2cm,left=2cm,right=2cm}

\title{КПЗ №1}
\author{Есиков Сергей Дмитриевич и Иванова Алиса Игоревна}
\date{\today}

\begin{document}
	\maketitle
	\subsection*{Задача\newline}
	
	Заданы параметры катушки (L, RL) и частота резонанса.
	
	Необходимо:
	\begin{enumerate}
		\item построить последовательный колебательный контур;
		\item исследовать его свойства в частотной области и в координатах времени;
		\item сопоставить данные анализа в координатах частоты, и времени;
		\item исследовать влияние внутреннего сопротивления источника энергии
		на колебательные процессы в контуре.
	\end{enumerate}

	\subsection*{Условие\newline}
	
	Был выбран вариант № 16, по которому контур имеет следующие параметры
	\begin{itemize}
		\item $L$ (индуктивность катушки) -- 470 мкГн
		\item $R_L$  ( сопротивление потерь катушки) -- 4 Ом
		\item $\omega$ (резонансная частота контура) -- 32 кГц
	\end{itemize}
		
	\subsection*{Схема\newline}
	\begin{figure}[H]
		\centering
		\begin{circuitikz}[scale=1.5, european]
			 % цепь
			\draw (0, 0) to[isourceAM, l=$e$] (0, 2) to[R=$R_g$] (2, 2) to[R=$R_L$] (4, 2) to[cute inductor, l=$L$] (4, 0) to[C=$C$] (2, 0) to[R=$R_C$] (0, 0);
			%стрелка напряжения
			\draw[->] (0.5, 1.5) -- (0.5, 0.5);
			\draw (0.5, 1) node[right] {$u_e$};
			%стрелка тока
			\draw[-latex] (2.5, 0.5) -- (2.1, 0.5);
			\draw [rounded corners=2mm] (1.5, 0.5) rectangle (3.5, 1.5);
			\draw (2.4, 0.5) node[above] {i};
		\end{circuitikz}
	\end{figure}
	
	\newpage
	\section{Выбор конденсатора.}
	Вычислите ёмкость конденсатора, обеспечивающую резонанс на заданной частоте. Руководствуясь рядом $Е24$, выберите ёмкость, подходящуюдля конденсатора, обеспечивающего резонанс $LC$-контура на частоте ближайшей заданной. Будем считать, что сопротивления потерь конденсатора и катушки одинаковые: $R_C$ = $R_L$.
	
	\subsection*{Мат. Часть. \newline}
	\subsubsection*{Обозначим}	
	\begin{itemize}
		\item Общее последовательное сопротивление -- $R = R_g + R_L + R_C = R_g + 2R_L$

		\item ЭДС на выходе источника -- $u_e(t) = \begin{cases} 0,& t < 0 \\ e,& t \ge 0 \\ \end{cases}$
		
		\item Ток, протекающий через контур -- $i(t)$
		
		\item $i(0) = 0$
	\end{itemize}
	
	\subsubsection*{Закон Кирхгофа}
	
	\[e(t) = u_R + u_L + u_C\]
	
	\subsubsection*{Компонентные соотношения \newline}
	\[u_R(t) = Ri(t)\]
	
	\[u_L(t) = L\frac{di(t)}{dt}\]
	
	\[u_C(t) = \frac{1}{C}\int_{0}^{t}i(\tau)d\tau + u_C(0)\]
	
	\subsubsection*{Подстановка}
	
	\[u_e(t) = L\frac{di(t)}{dt} + Ri(t) + \frac{1}{C}\int_{0}^{t}i(\tau)d\tau + u_C(0)\]
	
	\subsubsection*{Для t>0}
	A
	\[L\frac{di(t)}{dt} + Ri(t) + \frac{1}{C}\int_{0}^{t}i(\tau)d\tau + u_C(0) = e\]
	
	\subsubsection*{Дифференцируем по t}
	
	\[L\frac{d^2i(t)}{dt^2} + R\frac{di(t)}{dt} + \frac{1}{C}i(t) = 0\]
	
	\subsubsection*{Делим на L}
	
	\[\ddot i(t) + \frac{R}{L} \dot i(t) + \frac{1}{CL}i(t) = 0\]
	
	\subsubsection*{Обозначим}
	
	\[\omega_0 = \frac{1}{CL}\]
	
	\[\delta =  \frac{R}{L}\]
	
	\subsubsection*{Для уравнения}
	
	\[\ddot i(t) + \delta \dot i(t) + \omega_0i(t) = 0\]
	
	
	
\end{document}